\documentclass[12pt]{beamer}
\usepackage[utf8]{inputenc}
\usepackage[english]{babel}
\usepackage{listings}
\usepackage{tabu}
\usepackage{booktabs}

\beamertemplatenavigationsymbolsempty
\AtBeginSection[]
{
    \begin{frame}
    \frametitle{Table of contents}
    \tableofcontents[currentsection]
    \end{frame}
}
\newcommand{\gray}{\textcolor{gray}}
\newcommand{\bigoh}[1]{\mathcal{O}\left(#1\right)}
\newcommand{\constant}{\bigoh{1}}
\newcommand{\linear}{\bigoh{n}}

\definecolor{mygreen}{rgb}{0,0.6,0}
\definecolor{mygray}{rgb}{0.5,0.5,0.5}
\definecolor{mymauve}{rgb}{0.58,0,0.82}

\lstset{ %
  backgroundcolor=\color{white},   % choose the background color; you must add \usepackage{color} or \usepackage{xcolor}
  basicstyle=\tiny,        % the size of the fonts that are used for the code
  breakatwhitespace=false,         % sets if automatic breaks should only happen at whitespace
  breaklines=true,                 % sets automatic line breaking
  commentstyle=\color{mygreen},    % comment style
  deletekeywords={...},            % if you want to delete keywords from the given language
  escapeinside={\%*}{*)},          % if you want to add LaTeX within your code
  extendedchars=true,              % lets you use non-ASCII characters; for 8-bits encodings only, does not work with UTF-8
  frame=single,                    % adds a frame around the code
  keepspaces=true,                 % keeps spaces in text, useful for keeping indentation of code (possibly needs columns=flexible)
  keywordstyle=\color{blue},       % keyword style
  language=C++,                 % the language of the code
  morekeywords={*,...},            % if you want to add more keywords to the set
  numbers=left,                    % where to put the line-numbers; possible values are (none, left, right)
  numbersep=5pt,                   % how far the line-numbers are from the code
  numberstyle=\tiny\color{mygray}, % the style that is used for the line-numbers
  rulecolor=\color{black},         % if not set, the frame-color may be changed on line-breaks within not-black text (e.g. comments (green here))
  showspaces=false,                % show spaces everywhere adding particular underscores; it overrides 'showstringspaces'
  showstringspaces=false,          % underline spaces within strings only
  showtabs=false,                  % show tabs within strings adding particular underscores
  stepnumber=1,                    % the step between two line-numbers. If it's 1, each line will be numbered
  stringstyle=\color{mymauve},     % string literal style
  tabsize=2                      % sets default tabsize to 2 spaces
}

\title{Linear data structures}
\subtitle{Arrays, vectors, linked lists}
\author{beOI Training}
\institute{\includegraphics[height=12em]{../share/beoi-logo}}

\begin{document}

\frame{\titlepage}

\section{Arrays and variants}

\begin{frame}[fragile]
\frametitle{Array}
\begin{lstlisting}
#define MAX_N 10000
int tab[MAX_N];

int main() {
    tab[1234] = 100;
    tab[1234]; // 100
    tab[5678]; // 0
}
\end{lstlisting}
\begin{itemize}
\item Size fixed at compile time
\item Accessing any element: $\constant$
\item Tip: if declared outside a function, initialised to zero
\end{itemize}
\end{frame}

\begin{frame}[fragile]
\frametitle{Bitset}
C++ : \texttt{bitset} \\
Java : \texttt{BitSet}
\begin{lstlisting}
bitset<MAX_N> tab; // bool tab[MAX_N];
tab[1234] = true;

bitset<4> b1(string("1100")),
          b2(string("0101"));
b1 | b2; // 1101
b1 & b2; // 0100
b1 >> 1; // 0110
\end{lstlisting}
\begin{itemize}
\item Like an array of booleans
\item 8 times more compact
\item Bitwise operations 64 times faster
\item See manual for the list of operations
\end{itemize}
\end{frame}

\begin{frame}[fragile]
\frametitle{Dynamic array: logic}
{\setlength{\parskip}{.9em}
If out of space, double the capacity

\def\arraystretch{1.3}

\begin{tabu} to .175\textwidth {|X[c]|X[c]|}
\hline
1 & \\
\hline
\end{tabu}
\hfill Capacity = 2

\begin{tabu} to .175\textwidth {|X[c]|X[c]|}
\hline
1 & 2 \\
\hline
\end{tabu}

\begin{tabu} to .35\textwidth {|X[c]|X[c]|X[c]|X[c]|}
\hline
1 & 2 & 3 & \\
\hline
\end{tabu}
\hfill Capacity = 4

\begin{tabu} to .35\textwidth {|X[c]|X[c]|X[c]|X[c]|}
\hline
1 & 2 & 3 & 4 \\
\hline
\end{tabu}

\begin{tabu} to .7\textwidth {|X[c]|X[c]|X[c]|X[c]|X[c]|X[c]|X[c]|X[c]|}
\hline
1 & 2 & 3 & 4 & 5 & & & \\
\hline
\end{tabu}
\hfill Capacity = 8} %\setlength
\end{frame}

\begin{frame}[fragile]
\frametitle{Dynamic array: in practice}
C++ : \texttt{vector} \\
Java : \texttt{ArrayList<E>}

\begin{lstlisting}
vector<int> vec(8, -1); // initialize to -1
vec[5] += vec[2];       // -2
vec.push_back(5);
vec.push_back(19);
vec.pop_back();
vec.back();             // 5
\end{lstlisting}
\begin{itemize}
\item Size increases and decreases
\item Accessing any element: $\constant$
\item Adding/removing an element \textbf{at the end}: $\constant$
\item Adding/removing elsewhere: $\linear$
\end{itemize}
\end{frame}

\section{Linked lists}

\begin{frame}[fragile]
\frametitle{Linked list: concept}
Nodes bound by links (pointers)
\begin{figure}
\centering
\includegraphics[width=.8\textwidth]{img/singly-linked}
\end{figure}
\begin{itemize}
\item Every node knows where is the next one
\item Nodes are not necessarily next to each other in memory
\end{itemize}
\begin{lstlisting}
struct Node {
    int value;
    Node *next; // link (pointer)
};
\end{lstlisting}
\end{frame}

\begin{frame}[fragile]
\frametitle{Linked list: iterating over elements}
Start at the first node and follow the links
\begin{figure}
\centering
\includegraphics[width=.6\textwidth]{img/singly-linked}
\end{figure}
The link in the last node contains NULL:
\begin{lstlisting}
Node *cur = start;   // always keep the first node!
while (cur != NULL) {
    cur->value;      // access value
    cur = cur->next; // switch pointer to next
}
\end{lstlisting}
\end{frame}

\begin{frame}[fragile]
\frametitle{Linked list: adding elements}
Only two links have to be changed
\begin{figure}
\centering
\includegraphics[width=.9\textwidth]{img/insert-node}
\end{figure}
\begin{lstlisting}
void insertAfter(Node *node, Node *new_node) {
    new_node->next = node->next;
    node->next = new_node;
}
\end{lstlisting}
\end{frame}

\begin{frame}[fragile]
\frametitle{Linked list: removing elements}
Change the link and remove the node
\begin{figure}
\centering
\includegraphics[width=.6\textwidth]{img/remove-node}
\end{figure}
\begin{lstlisting}
void removeAfter(Node *node) {
    Node *toRemove = node->next;
    node->next = node->next->next; // bypass
    deleted toRemove;
}
\end{lstlisting}
\end{frame}

\begin{frame}[fragile]
\frametitle{Linked list: limitations}
\begin{figure}
\centering
\includegraphics[width=.8\textwidth]{img/singly-linked}
\end{figure}
With a (singly) linked list:
\begin{itemize}
\item Adding/removing at the beginning: $\constant$
\item Adding/removing at a \textbf{given} location: $\constant$
\end{itemize}
Operations at the back:
\begin{itemize}
\item Adding at the back: $\constant$
\item Removing at the back: impossible to do directly, $\linear$
\end{itemize}
\end{frame}

\begin{frame}[fragile]
\frametitle{Doubly linked list}
Links in both ways!
\begin{figure}
\centering
\includegraphics[width=.8\textwidth]{img/doubly-linked}
\end{figure}
\begin{itemize}
\item Iteration in both ways
\item Removing at the back $\constant$
\item A bit more memory-heavy
\end{itemize}
\begin{lstlisting}
struct Node {
    int value;
    Node *prev, *next; // two pointers
};
\end{lstlisting}
\end{frame}

\begin{frame}[fragile]
\frametitle{Linked lists: in practice}
C++ : \texttt{list} \\
Java : \texttt{LinkedList<E>}
\begin{lstlisting}
list<int> l;
list<int>::iterator it;

l.push_back(3);  // 3
it = l.begin();  // ^ points to 3
l.push_back(4);  // 3 4
l.push_front(1); // 1 3 4
l.insert(it, 2); // 1 2 3 4 (inserts before 3)
l.pop_front();   // 2 3 4
l.pop_back();    // 2 3
\end{lstlisting}
\begin{itemize}
\item \texttt{list<>} is doubly linked
\item Remember positions using \texttt{iterators}
\item Everything is $\constant$
\end{itemize}
\end{frame}

\section{Queue and stack}

\begin{frame}
\frametitle{Queue: concept}
\begin{itemize}
\item Queuing in a store
\item We add at the back, we remove at the front
\item "First In First Out"
\end{itemize}
\begin{figure}
\centering
\includegraphics[width=.6\textwidth]{img/queue}
\end{figure}
\end{frame}

\begin{frame}[fragile]
\frametitle{Queue: in practice}
C++ : \texttt{queue} \\
Java : \texttt{Queue<E>}
\begin{itemize}
\item Adding at the back, removing at the front $\Rightarrow$ linked list
\item Everything is $\constant$
\end{itemize}
\begin{lstlisting}
queue<int> q;
q.push(1);
q.push(2);
q.front(); // 1
q.pop();
q.front(); // 2
\end{lstlisting}
\end{frame}

\begin{frame}
\frametitle{Stack: concept}
\begin{itemize}
\item Stack of pancakes
\item We add at the top, we remove from the top
\item "Last In First Out"
\end{itemize}
\begin{figure}
\centering
\includegraphics[width=.5\textwidth]{img/stack}
\end{figure}
\end{frame}

\begin{frame}[fragile]
\frametitle{Stack: in practice}
C++ : \texttt{stack} \\
Java : \texttt{Stack<E>}
\begin{itemize}
\item Adding and removing at the back $\Rightarrow$ liste chaînée \emph{ou vecteur}
\item Everything is $\constant$
\end{itemize}
\begin{lstlisting}
stack<int> q;
q.push(1);
q.push(2);
q.top(); // 2
q.pop();
q.top(); // 1
\end{lstlisting}
\end{frame}

\section{Choosing the right structure}

\begin{frame}
\frametitle{Choice: special structures}
Structures for special needs:
\begin{itemize}
\item Adding from one side and removing from the other $\Rightarrow$ \textbf{queue}
\item Adding and removing from the same side $\Rightarrow$ \textbf{stack}
\item Booleans, bitwise operations (and, or, shift, ...) $\Rightarrow$ \textbf{bitset}
\end{itemize}

~

Otherwise, see next slide!
\end{frame}

\begin{frame}
\frametitle{Choice: arrays, vectors, linked lists}
``Add'' = adding or removing
\begin{center}
\begin{tabu}{l|cccc}
\toprule
Structure & Indexation & Add (back) & Add (middle) \\
\midrule
Array & $\constant$ & $\linear$ & $\linear$ \\
Vector & $\constant$ & $\constant$ & $\linear$ \\
Linked list & $\linear$ & $\constant$ & $\constant$ \\
\bottomrule
\end{tabu}
\end{center}
\begin{itemize}
\item Adding in the middle (rare) $\Rightarrow$ \textbf{list}
\item Unknown maximal size $\Rightarrow$ \textbf{vector}
\item All other cases $\Rightarrow$ \textbf{array} (faster)
\end{itemize}
\end{frame}

\begin{frame}
\frametitle{Sources of the images}
\begin{itemize}
\item \url{https://commons.wikimedia.org/wiki/File:Singly-linked-list.svg}
\item \url{https://commons.wikimedia.org/wiki/File:CPT-LinkedLists-addingnode.svg}
\item \url{https://en.wikipedia.org/wiki/File:CPT-LinkedLists-deletingnode.svg}
\item \url{https://en.wikipedia.org/wiki/File:Doubly-linked-list.svg}
\item \url{https://en.wikipedia.org/wiki/File:Data_Queue.svg}
\item \url{https://en.wikipedia.org/wiki/File:Data_stack.svg}
\end{itemize}
\end{frame}

\end{document}
