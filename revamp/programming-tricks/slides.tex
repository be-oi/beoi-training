\documentclass[handout,code=programming-tricks,title={Efficient
  implementation and debugging}]{../share/cpslide}

\slideinit

\begin{document}

\slidebegin

\section{Why programming tricks?}

\begin{frame}{Why programming tricks?}
  \begin{itemize}
  \item During contests, you will have to solve a lot of subtasks when you
    cannot solve the full problems.
  \item You have to be able to do this efficiently because you will need as much
    time as possible to solve other problems/subtasks.
  \item Here are a few tricks to help you solve problems efficiently.
  \end{itemize}
\end{frame}

\section{Bit-level operators}

\begin{frame}{Bit-level operators}
 As you probably know, computers represent numbers in base two, that is with \ci{0}s
 and \ci{1}s. Some C++ operators are designed to make use of this. There are shift
 operators to quickly multiply or divide by powers of 2 and bitwise operators to
 act on the bits of a number separately.
\end{frame}

\begin{frame}{Bitshit operators}
  There are two bitshift operators.
  \begin{itemize}
  \item \ci{a\textless{}\textless b} Shift $a$, $b$ bits to the left. This means that
    $a$ is basically divided by $2^b$ and then rounded down.
  \item \ci{a\textgreater{}\textgreater b} Shift $a$, $b$ bits to the right. This means that
    $a$ is basically multiplied by $2^b$.
  \end{itemize}

  \begin{warning}
    Type limits apply here. Also, if $b$ is bigger than the number of bits of the
    type, these two operations are not defined anymore, which means they could
    return an arbitrary result. Be very careful if you are using variables as
    shift parameters.
  \end{warning}
\end{frame}

\begin{frame}{Bitwise operators}
  There are four bitwise operators.
  \begin{itemize}
  \item \ci{\textasciitilde a} Bitwise NOT. 1 if the input is 0. 0 if the input is 1.
  \item \ci{a|b} Bitwise OR. 1 if at least one of the inputs is 1. 0 if both
    inputs are 0.
  \item \ci{a\^{}b} Bitwise XOR. 1 if both inputs are not equal. 0 if both
    inputs are equal.
  \item \ci{a\&b} Bitwise AND. 1 if both inputs are 1. 0 if at least one of the
    inputs is 1.
  \end{itemize}
\end{frame}

\begin{frame}{Bit-level operators}
  \begin{warning}
   Bit-level operators have a very low priority compared to other operators, which means you
   should put them inside parentheses or even create separate functions for
   them. For example, \ci{a\&1 == 0} is actually equivalent to \ci{a\&(1 == 0)},
   which is always \ci{false}.

   There is a difference between \ci{\&|} and \ci{\&\&||} for \ci{bool}s. The
   former always evaluate both operands while the latter \textit{short circuit}:
   they only evaluate the right operand if necessary. This allows you to write
   \ci{if(i < n \&\& arr1[i] < arr2[i])}, for example.
  \end{warning}
\end{frame}

\begin{frame}{Iterate over all subsets}
  Use a for loop to iterate over all $2^n$ subsets.
  \src{allsubsets.cpp}

  \pause
  \begin{question}{How do we exclude the empty subset? ($\{\}$)}
    We start the loop at $1$.
  \end{question}
\end{frame}

\begin{frame}{Iterate over all permutations}
  Use \ci{next\_permutation()} to iterate over all $n!$ permutations.
  \src{allpermutations.cpp}

  \begin{question}{How do we iterate over all combinations (pick $k$ elements among $n$)?}
    Use permutations of $k$ \ci{1}s and $n-k$ \ci{0}s.
  \end{question}
\end{frame}

\section{Debugging}

\begin{frame}{Debugging}
  Why use a debugger (gdb)?
  \begin{itemize}
  \item It allows you to locate where the program segfaults.
  \item It allows you to print any global and local variable, in any function
    call.
  \item It allows you to set breakpoints to inspect variables at any point in
    time.
  \item Much more...
  \end{itemize}
  Learn to use it now!
\end{frame}

\begin{frame}{Debugging}
  How to setup a debugger?
  \begin{itemize}
  \item Install \ci{gdb} if you don't have it already. Do it now!
  \item Add a \ci{-g} flag to your compilation command. For example,
    \ci{g++ -std=c++14 -g main.cpp}.
  \item Open \ci{gdb}
  \end{itemize}
\end{frame}

\begin{frame}{Gdb commands}
  \begin{info}{Common gdb commands}
    \begin{tabular}{ll}
      \ci{file <name>} & Load file \\
      \ci{r} & Run program \\
      \ci{\^{}C} & Terminate program \\
      \ci{q} & Quit gdb \\
      \\
      \ci{print <x>} & Print variable \\
      \ci{bt} & Print backtrace \\
      \ci{up} & Move one frame (one function call) up \\
      \ci{down} & Move one frame (one function call) down
    \end{tabular}
  \end{info}
  Download \ci{examplecode.cpp} and try these functions now.
\end{frame}

\begin{frame}{Effective debugging}
  Great, now you know how to inspect your program? What to do if your program
  doesn't behave as expected, be it WA on a sample or on the server.
  \begin{itemize}
  \item Step 1: Always, always find a test case that breaks your code. As long
    as you didn't prove that your code isn't correct by building a killer test
    case, you cannot check that the bug disappeared. I like to name the test
    cases \ci{x.in} and \ci{x.out} where \ci{x} is the test case number.
  \item Step 2: Fix the bug by using gdb to find where the code fails.
  \item Step 3: Test all the test cases you have so far to make sure you did not
    reintroduce a previous bug.
  \item Repeat until you get AC.
  \end{itemize}
\end{frame}

\begin{frame}{Effective debugging}
  What should you do if you cannot find a killer test case or if the test cases
  are so big that you have no way to compute the result by hand?
  \begin{itemize}
  \item Read the problem statement once again.
  \item Write a bruteforce to test the code (maybe you already have one from a
    previous subtask).
  \item If you don't have the time to write a bruteforce, try to find invariants
    in the input and change the input accordingly. For example, the order of the
    elements for the knapsack problem is irrelevant, so you could try
    permutating the elements in the input.\footnote{Credit goes to errichto.}
  \end{itemize}
\end{frame}

\section{Extra tips}
\begin{frame}{Editor choice}
  \begin{itemize}
  \item Learn to use Vim now and use it for competitive programming.
  \item Vim is not an IDE, but it's one of the best text editors.
  \item Vim has incredibly short keybindings for a lot of editing operations.
  \item The learning curve is extremely steep, but it's totally worth it. Try
    not to use arrow keys. \textit{Not even in insert mode.}
  \item (Bonus) When you will need an IDE and not a text editor, you will be
    ready to use Spacemacs!
  \end{itemize}
\end{frame}

\begin{frame}{Templates}
  Use a template. Write it once at the beginning of the contest and reuse it for
  every single problem. Now, you may be wondering what uses there are for a template.
  \begin{info}{Killer features}
    \begin{itemize}
    \item Macros help you prevent stupid errors like
      \ci{for(int i = 0; i < n; ++j)} while you code.
    \item Macros allow you to type less code to get the same result. Compare
      \ci{V\textless{}V\textless{}int\textgreater{}\textgreater} and
      \ci{vector\textless{}vector\textless{}int\textgreater{}\textgreater}.
    \item Macros provide a consistent way to print debug output, combined with a
      global switch to disable them.
    \end{itemize}
  \end{info}
\end{frame}

\begin{frame}{Example template}
  \src{template.cpp}
\end{frame}

\begin{frame}{Aliases}
  Always use an alias to compile. An alias is a shortcut to some command. You'll
  probably want to use \ci{-fsanitize=address,undefined} to quickly discover
  issues with your code, but then you'll want another alias for gdb specifically.
  Put the following in your \ci{.bashrc}:
  \lstinputlisting[language=bash]{src/alias.sh}
  Usage: \ci{g filename.cpp}.
\end{frame}
\end{document}
