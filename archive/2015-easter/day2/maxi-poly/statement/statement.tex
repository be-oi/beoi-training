\documentclass[a4paper,12pt]{article}

\usepackage{mystyle}

\lstset
{
    basicstyle=\ttfamily,
    frame=single,
}

\begin{document}

\section*{Maximum term in a polynomial}

You are given a polynomial of degree $N-1$, with $N$ terms ($1 \leq N \leq 10^5$) with indeterminate $x$ in the reals.
All coefficients in the polynomial are positive.
The problem is to find which terms of the polynomial will be the largest term (strictly larger than all the others) for at least one positive value of $x$.

Let's take an example: the polynomial $1 + 5x + x^2$.
\begin{itemize}
    \item for $0 < x < 1/5$, the term $1$ is larger than $5x$ and $x^2$;
    \item for $1/5 < x < 5$, the term $5x$ is larger than $1$ and $x^2$;
    \item for $5 < x$, the term $x^2$ is larger than $1$ and $5x$;
\end{itemize}
so here, all three terms are the largest term for some positive value of $x$.

But if we take $1 + x/5 + x^2$, we notice that $x/5$ is never the largest term.
Indeed, it would be larger than $1$ for $x > 5$ and larger than $x^2$ for $x < 1/5$.
There is no intersection. Instead:
\begin{itemize}
    \item for $0 < x < 1$, the term $1$ is larger than $x/5$ and $x^2$;
    \item for $1 < x$, the term $x^2$ is larger than $1$ and $x/5$;
\end{itemize}
so here, only $1$ and $x^2$ are the largest term for some positive value of $x$.

You are asked to give the degree of each term that is the largest term at some point.
For the first example, you would give 0, 1 and 2, since $1$, $5x$ and $x^2$ have degrees 0, 1 and 2 (in that order).
For the second example, you would give only 0 and 2, since $1$ and $x^2$ have degrees 0 and 2 (in that order).

No term will be the non-strictly largest term without being the strictly largest term for some non-empty open interval.
So such situations as $1 + x + x^2$, where $x$ is the largest term for $x=1$, in a tie with $1$ and $x^2$, and lower than either $1$ or $x^2$ for all other values, will never be included in the input.

\subsection*{Input}

\begin{itemize}
    \item The input starts with one line containing the integer $T$, the number of test cases to follow.
    \item Each test case consists of two lines.
    \item The first line of a test case contains the integer $N$, the number of terms in the polynomial. So the degree of the polynomial will be $N-1$.
    \item The second line of a test case contains $N$ space-separated reals, the coefficients of the polynomial, by increasing degree. So the first coefficient will be the constant term, the second coefficient the term in $x$, etc. and the last coefficient the term in $x^{N-1}$. The coefficients may be in scientific notation, which your language will surely process without any trouble.
\end{itemize}

\subsection*{Output}

The output must consist of $T$ lines, one for each test case.
On each line, print a space-separated list of all the degrees of the terms which are the largest term for some positive value of $x$, in increasing order.

\subsection*{Sample input}

\lstinputlisting{../test-cases/small.in}

\subsection*{Sample output}

\lstinputlisting{../test-cases/small.out}

\vspace{1em}
{\footnotesize Problem setter: Victor Lecomte}

\end{document}
