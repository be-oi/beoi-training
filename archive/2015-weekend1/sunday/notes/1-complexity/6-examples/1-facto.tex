\subsubsection{Factorization}

Factorizing a single integer, or testing if it is prime, can be done
using this algorithm:

\begin{verbatim}
vector<int> factorize(int n)
{
    vector<int> f;
    for(int i=2; i*i <= n; i++)
    {
        while(n%i == 0)
        {
            f.push_back(i);
            n /= i;
        }
    }
    if(n != 1)
        f.push_back(n);
    return f;
}
\end{verbatim}

The trick here is to check divisors up to $\sqrt{n}$.
If $n$ is divisible by some factor $i > \sqrt{n}$, then it must also
be divisible by $n/i < \sqrt{n}$, which has been detected already.

If at the end of the loop $n$ is not 1, it means the remaining value is a prime,
and we add it to the factors.

Since $i$ can only take values 2 through $\sqrt{n}$,
and $n$ can only have $\log_2n$ prime divisors,
this algorithm has a worst-case running time of $O(\sqrt{n})$.

However, if you need to factorise or test for prime many integers, the Sieve of
Eratosthenes is much more efficient.

Let us consider you want to factorise integers in the range $[1,n)$.
The idea is to precompute the largest prime factor of every integer,
so that by looking at smaller and smaller divisors of $n$ you can factorize
it completely in $O(\log n)$ time.

This is the precomputation step, assuming $n=10^6$:

\begin{verbatim}
int n = 1000000;
int factor[1000000];

void precompute_factors()
{
    for(int i=2; i<n; i++)
    {
        if(factor[i] == 0)
        {
            for(int j=1; i*j < n; j++)
            {
                factor[i*j] = i;
            }
        }
    }
}
\end{verbatim}

(This implementation uses the fact that arrays declared
outside of functions
are initialized with zeroes.)

And this is the factorizing function:

\begin{verbatim}
vector<int> factorize(int a)
{
    vector<int> f;
    while(a != 1)
    {
        f.push_back(factor[a]);
        a /= factor[a];
    }
    return f;
}
\end{verbatim}

For obscure reasons linked to the density of primes, the precomputation
runs in $O(n \log(\log n))$ time, which is very nearly linear time,
while the actual factorizing function runs in logarithmic time.
