\subsubsection{Euclid's algorithm}

The following function, called Euclid's algorithm,
gives the GCD (greatest common divisor)
of positive integers $a$ and $b$:

\begin{verbatim}
int gcd(int a, int b)
{
    if(b == 0)
        return a;
    else
        return gcd(b, a%b);
}
\end{verbatim}

Proving that the algorithm is correct requires some knowledge of arithmetic.
We will only determine the complexity here.

The usual tricks are uneffective.
There is no loop, and the number of times the function will call itself depends
on $a$ and $b$.

However, we notice that in the \texttt{gcd(b, a\%b)} call, $b$ is always
bigger than $a\mod b$.
Thus, we can assume $a > b$ after the first recursive call.

There are two cases:
\begin{itemize}
    \item If $b > a/2$, then $a\mod b = a-b < a/2$.
    \item If $b \leq a/2$, then $a\mod b < b \leq a/2$,
        by the definition of modulo.
\end{itemize}

We see that in both cases, $a\mod b < a/2$. Since $a\mod b$ is the $a$-parameter
of the function two calls later, that parameter is divided by two every two
calls.
This implies that after at most $2\log_2 n$ recursive calls,
$a$ will drop to zero.
But it cannot do so because of the \texttt{if} clause.

From this we conclude the function will terminate after less
than $2\log_2 n$ calls, so it has $O(\log n)$ time complexity.
