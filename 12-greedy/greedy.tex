\documentclass[12pt]{beamer}
\usepackage{listings}
\usepackage[]{color}
\usepackage{bbding}
\usepackage{ragged2e}
\usepackage{tikz}
\usetikzlibrary{decorations.pathreplacing}

\beamertemplatenavigationsymbolsempty
\AtBeginSection[]
{
    \begin{frame}
    \frametitle{Table of Contents}
    \tableofcontents[currentsection]
    \end{frame}
}
\setlength{\tabcolsep}{10pt}
\newcommand{\bigoh}[1]{\mathcal{O}\left(#1\right)}
\newcommand{\TLE}{\textcolor{blue}{TLE}}
\newcommand{\WA}{\textcolor{red}{WA}}
\newcommand{\MLE}{\textcolor{orange}{MLE}}
\newcommand{\AC}{\textcolor{green}{AC}}
\newcommand{\blank}{\vspace{.5cm}}

\definecolor{mygreen}{rgb}{0,0.6,0}
\definecolor{mygray}{rgb}{0.5,0.5,0.5}
\definecolor{mymauve}{rgb}{0.58,0,0.82}

\lstset{ %
  backgroundcolor=\color{white},   % choose the background color; you must add \usepackage{color} or \usepackage{xcolor}
  basicstyle=\tiny,        % the size of the fonts that are used for the code
  breakatwhitespace=false,         % sets if automatic breaks should only happen at whitespace
  breaklines=true,                 % sets automatic line breaking
  commentstyle=\color{mygreen},    % comment style
  deletekeywords={...},            % if you want to delete keywords from the given language
  escapeinside={\%*}{*)},          % if you want to add LaTeX within your code
  extendedchars=true,              % lets you use non-ASCII characters; for 8-bits encodings only, does not work with UTF-8
  frame=single,                    % adds a frame around the code
  keepspaces=true,                 % keeps spaces in text, useful for keeping indentation of code (possibly needs columns=flexible)
  keywordstyle=\color{blue},       % keyword style
  language=C++,                 % the language of the code
  morekeywords={*,...},            % if you want to add more keywords to the set
  numbers=left,                    % where to put the line-numbers; possible values are (none, left, right)
  numbersep=5pt,                   % how far the line-numbers are from the code
  numberstyle=\tiny\color{mygray}, % the style that is used for the line-numbers
  rulecolor=\color{black},         % if not set, the frame-color may be changed on line-breaks within not-black text (e.g. comments (green here))
  showspaces=false,                % show spaces everywhere adding particular underscores; it overrides 'showstringspaces'
  showstringspaces=false,          % underline spaces within strings only
  showtabs=false,                  % show tabs within strings adding particular underscores
  stepnumber=1,                    % the step between two line-numbers. If it's 1, each line will be numbered
  stringstyle=\color{mymauve},     % string literal style
  tabsize=2                      % sets default tabsize to 2 spaces
}

\title{Greedy}
\subtitle{Greedy is good}
\author{beOI Training }
\institute{\includegraphics[height=12em]{../share/beoi-logo}}
\date{} % No date
\begin{document}

\frame{\titlepage}

\section{Greedy traits}

\begin{frame}
    "The point is, ladies and gentleman, that 'greed', for lack of a better word, is good."\\
    
    \textit{Gordon Gecko, Wall Street}
    

\end{frame}

\begin{frame}
    \frametitle{Traits of a greedy person}
    A greedy person
    \begin{itemize}
    	\item Doesn't care about the future
    	\item Doesn't dwell on the past
    	\item Looks only at the present situation
    	\item Takes the biggest/best thing currently available
    	
    \end{itemize}
\end{frame}

\begin{frame}
	\frametitle{Traits of a greedy algorithm}
	A greedy algorithm
	\begin{itemize}
		\item Makes the locally optimal choice at any state.
		\item Doesn't know anything about a future state.
		\item Doesn't go back for fixing mistakes.
		
	\end{itemize}
\end{frame}


\section{Example problems}

\begin{frame}
	\frametitle{A shortest path algorithm}
	\pause
	At every point take the shortest edge and go from there, until you get to the destination.\\\blank
	\pause
	Will this work? Counterexample?\\\blank
	\pause
	No! This is not an algorithm, but a \textbf{heuristic} (use Dijkstra)
\end{frame}

\begin{frame}
	\frametitle{Coin change}
	You have a given set of coin types (ex: $\{25,10,5,1\}$)\\
	We have an unlimited amount of coins.\\
	How can we give a certain amount of money with the least amount of coins?\\
	Example: Give 42 cents back
	\\\blank Does the greedy algorithm work for every coin set? Counterexample\\\blank
	Try making 6 cents with {4,3,1}
\end{frame}

\begin{frame}
	\frametitle{Does it ever work?}
	\pause
	... seems like it doesn't\vfill
	\pause
	But sometimes it does!
\end{frame}

\begin{frame}
	\frametitle{Interval scheduling}
	A set of activities, each with a starting and ending time.\\
	How can we schedule the most number of activities?\\\blank
	
	Let's try some ideas:
	\begin{enumerate}
		\item Earliest starting time? \pause \color{red}No!\color{black}\pause
		\item Shortest interval? \pause \color{red}No!\color{black}\pause
		\item Earliest ending time? \pause \color{green}Yes!\color{black}
	\end{enumerate}
\end{frame}

\begin{frame}
	\frametitle{Load balancing}
	Certain number of containers $C$.\\
	Certain number of items $S$ with a certain mass $M_i$.\\
	$1 \leq S \leq 2C$\\
	Minimize inbalance:\\ 
	$$A = \frac{\sum_{j=1}^{S}M_j}{C}, 
	Imbalance = \sum_{i=1}^{C} |X_i - A|$$
	where $X_i$ is the total mass in chamber $i$
\end{frame}

\begin{frame}
	\frametitle{Load balancing}
	Any idea?\\\blank\pause
	Here's a hint: make sure there are exactly $2C$ items by adding dummy elements.\\\blank\pause
	Sort the items and pair heaviest with the lightest.\\\blank
	Can you prove this works?
\end{frame}


\section{General remarks}

\begin{frame}
	\frametitle{General remarks}
	\begin{itemize}
		\item Every greedy algorithm has the \textbf{greedy choice property}(Reach global optimum from local optimum)\\ and the \textbf{optimal substructure property}(Optimal solution to subproblems $=>$ optimal solution to problem)\\
		\pause
		\item Hard to prove, easy to code $=>$ just try it (or find a counterexample)
	\end{itemize}
\end{frame}




\end{document}
