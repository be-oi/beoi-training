\documentclass[12pt]{beamer}
\usepackage{listings}
%\usepackage{tabu}
%\usepackage{booktabs}
\beamertemplatenavigationsymbolsempty
\AtBeginSection[]
{
    \begin{frame}
    \frametitle{Table of Contents}
    \tableofcontents[currentsection]
    \end{frame}
}
\lstset{language=C++, basicstyle=\footnotesize, frame=single}

\title{Number theory}
\subtitle{Prime factors, sieve, GCD, extended Euclid}
\author{beCP Training}
\institute{\includegraphics[height=12em]{../share/beoi-logo}}

\begin{document}

\frame{\titlepage}

\section{Prime check}

\begin{frame}
\frametitle{Prime numbers}
\end{frame}

\begin{frame}
\frametitle{Prime check: linear}
\end{frame}

\begin{frame}
\frametitle{Prime check: stop at square root}
\end{frame}

\section{Sieve of Eratosthenes}

\begin{frame}
\frametitle{Sieve idea}
\end{frame}

\begin{frame}
\frametitle{Sieve example}
\end{frame}

\begin{frame}
\frametitle{Sieve implementation}
\end{frame}

\begin{frame}
\frametitle{Prime check with sieve}
\end{frame}

\begin{frame}
\frametitle{Sieve by-products}
\end{frame}

\section{Greatest Common Divisor}

\begin{frame}
\frametitle{GCD definition}
\end{frame}

\begin{frame}
\frametitle{Euclid for GCD}
\end{frame}

\begin{frame}
\frametitle{GCD implementation}
\end{frame}

\begin{frame}
\frametitle{Least Common Multiple}
\end{frame}

\section{Extended Euclid}

\begin{frame}
\frametitle{Linear diophantine equation}
\end{frame}

\begin{frame}
\frametitle{Building the solutions}
\end{frame}

\begin{frame}
\frametitle{Extended Euclid}
\end{frame}

\begin{frame}
\frametitle{Bonus property}
\end{frame}

\end{document}
