\documentclass[12pt]{beamer}
\usepackage[utf8]{inputenc}
\usepackage[frenchb]{babel}
\usepackage{listings}
\usepackage{tabu}
\usepackage{color}
\usepackage{booktabs}
\beamertemplatenavigationsymbolsempty
\AtBeginSection[]
{
    \begin{frame}
    \frametitle{Table of Contents}
    \tableofcontents[currentsection]
    \end{frame}
}
\lstset{language=C++, basicstyle=\footnotesize, frame=single}
\newcommand{\gray}{\textcolor{gray}}

\title{Algorithmes et complexité}
\subtitle{Définitions, notation O}
\author{Training beOI}
\institute{\includegraphics[height=12em]{../shared-img/beoi-logo}}

\begin{document}

\frame{\titlepage}

\section{Algorithmes}

\begin{frame}
\frametitle{Qu'est-ce qu'un algorithme?}
\begin{itemize}
\item Une manière de calculer un résultat
\item Une idée pour résoudre un problème
\item Une suite d'instructions
\item La description d'un programme
\end{itemize}
\end{frame}

\begin{frame}
\frametitle{Qu'est qu'un \emph{bon} algorithme?}
Domaine du programmeur:
\begin{itemize}
\item Ne crashe pas
\item Se termine
\item Donne la bonne réponse
\end{itemize}

~

Notre domaine:
\begin{itemize}
\item Est rapide
\item Utilise peu de mémoire
\item \textbf{Est accepté en concours}
\end{itemize}
\end{frame}

\section{Complexité}

\begin{frame}
\frametitle{Mesurer l'efficacité}
Idées:
\begin{itemize}
\item Chronométrer
\item Mesurer la RAM
\end{itemize}

~

Mais varie selon:
\begin{itemize}
\item Le langage
\item L'implémentation
\item La machine
\item L'heure de la journée
\end{itemize}
\end{frame}

\begin{frame}
\frametitle{La notation O}
Trouver une notion d'efficacité \emph{intrinsèque}:
\begin{itemize}
\item Faisons grandir l'input
\item Regardons comment la vitesse évolue
\end{itemize}

~

Exemple: calculer $1+\cdots+n$. Si $n$ multiplié par 2, le temps d'exécution:
\begin{itemize}
\item Reste constant: $O(1)$
\item Est multiplié par 2: $O(n)$
\item Est multiplié par 4: $O(n^2)$
\end{itemize}
\end{frame}

\begin{frame}[fragile]
\frametitle{Temps constant}
\textbf{Problème:} calculer la somme $1+2+\cdots+n$.

~

\textbf{Solution 1:} Un simple calcul
\begin{lstlisting}
int sum = n * (n-1) / 2;
\end{lstlisting}
\begin{itemize}
\item Temps ne change pas si $n$ double
\item Temps ``constant''
\item Complexité $O(1)$
\item Temps ``proportionnel à $1$''
\end{itemize}
\end{frame}

\begin{frame}[fragile]
\frametitle{Temps linéaire}
\textbf{Solution 2:} Une boucle
\begin{lstlisting}
int sum = 0;
for (int i = 1; i <= n; i++)
    sum += i;
\end{lstlisting}
\begin{itemize}
\item Temps double si $n$ double
\item Temps ``linéaire''
\item Complexité $O(n)$
\item Temps ``proportionnel à $n$''
\end{itemize}
\end{frame}

\begin{frame}[fragile]
\frametitle{Temps quadratique}
\textbf{Solution 3:} Deux boucles (stupide!)
\begin{lstlisting}
int sum = 0;
for (int i = 1; i <= n; i++)
    for (int j = 1; j <= i; j++)
        sum++;
\end{lstlisting}
\begin{itemize}
\item Temps quadruple si $n$ double
\item Temps ``quadratique''
\item Complexité $O(n^2)$
\item Temps ``proportionnel à $n^2$''
\end{itemize}
\end{frame}

\begin{frame}
\frametitle{Puissances}
Définition:
\begin{itemize}
\item Multiplications en chaîne
\item ``3 exposant $n$''
\item $3^n = \underbrace{3\times\cdots\times3}_\text{$n$ fois}$
\end{itemize}
Exemples:
\begin{itemize}
\item $3^0 = 1$ (par définition)
\item $3^1 = 3$
\item $3^2 = 3 \times 3 = 9$ (carré)
\item $3^3 = 3 \times 3 \times 3 = 27$ (cube)
\end{itemize}
\end{frame}

\begin{frame}
\frametitle{Logarithmes: intuition (1)}
Jeu à deux joueurs:
\begin{itemize}
\item Alice choisit entre 1 et 16
\item À chaque tour:
    \begin{itemize}
    \item Bob donne à Alice un ou plusieurs nombres
    \item Alice dit si le nombre est parmi ceux-là 
    \end{itemize}
\item Quand Bob trouve le nombre il a gagné
\item Comment gagner en peu de tours?
\end{itemize}
\end{frame}

\begin{frame}
\frametitle{Logarithmes: intuition (2)}
\textbf{Stratégie:} Donner la moitié des nombres possibles
\begin{itemize}
\item D'abord 8 nombres parmi les 16
\item Puis 4 parmi les 8 restants
\item Puis 2 parmi les 4 restants
\item Puis 1 des 2 restants
\item Trouvé!
\end{itemize}

~

Donc 4 questions suffisent.
\end{frame}

\begin{frame}
\frametitle{Logarithmes: intuition (3)}
De manière générale, avec $n$ nombres au départ, combien de questions?

~

Combien de fois peut-on couper en deux?
\begin{itemize}
\item Si $n=2$, une fois
\item Si $n=4$, deux fois
\item Si $n=8$, trois fois
\item Si $n=16$, quatre fois
\end{itemize}
\end{frame}

\begin{frame}
\frametitle{Logarithme en base 2}
La fonction qui répond à cette question est $\log_2$: le logarithme en base 2. Par exemple
\begin{itemize}
\item $\log_2 (2) = 1$
\item $\log_2 (4) = 2$
\item $\log_2 (8) = 3$
\item $\log_2 (16) = 4$
\end{itemize}
La stratégie de Bob est en $O(\log_2(n)) = O(\log n)$.

~

En fait, le logarithme, c'est l'exposant qu'il faut mettre à 2 pour atteindre $n$:
\[ x = \log_2 (n)\ \Leftrightarrow\ 2^x = n \]
\end{frame}

\begin{frame}
\frametitle{Logarithme général (bonus)}
Pas seulement pour deux! Le logarithme en base $a$, c'est le nombre de fois qu'on peut diviser par $a$. Par exemple:
\begin{itemize}
\item $\log_3(27) = 3$
\item $\log_4(16) = 2$
\item $\log_5(5) = 1$
\end{itemize}

~

En fait, c'est l'exposant qu'il faut mettre à $a$ pour atteindre $n$:
\[ x = \log_a (n)\ \Leftrightarrow\ a^x = n \]
C'est un peu ``l'inverse'' des puissances.
\end{frame}

\begin{frame}[fragile]
\frametitle{Recherche dans un tableau trié (1)}
On nous donne un tableau dans l'ordre croissant:
\begin{center}
\def\arraystretch{1.3}
\begin{tabu} to .7\textwidth {|X[c]|X[c]|X[c]|X[c]|X[c]|X[c]|X[c]|}
\hline
1 & 4 & 6 & 9 & 15 & 23 & 24 \\
\hline
\end{tabu}
\end{center}
Vérifier si un nombre $x$ s'y trouve.

~

\textbf{Solution 1:} Tout parcourir, linéaire $O(n)$
\begin{lstlisting}
bool isIn(int tab[], int n, int x)
{
    for (int i = 0; i < n; i++)
        if (tab[i] == x)
            return true;
    return false;
}
\end{lstlisting}
\end{frame}

\begin{frame}
\frametitle{Recherche dans un tableau trié (2)}
On cherche 7.

~

\textbf{Idée:} regarder au milieu et comparer:
\begin{center}
\def\arraystretch{1.3}
\begin{tabu} to .7\textwidth {|X[c]|X[c]|X[c]|X[c]|X[c]|X[c]|X[c]|}
\hline
1 & 4 & 6 & \textbf{9} & 15 & 23 & 24 \\
\hline
\end{tabu}
\end{center}
Trop grand ($9>7$), allons à gauche:
\begin{center}
\def\arraystretch{1.3}
\begin{tabu} to .7\textwidth {|X[c]|X[c]|X[c]|X[c]|X[c]|X[c]|X[c]|}
\hline
1 & \textbf{4} & 6 & \gray{9} & \gray{15} & \gray{23} & \gray{24} \\
\hline
\end{tabu}
\end{center}
Trop petit ($4<7$), allons à droite:
\begin{center}
\def\arraystretch{1.3}
\begin{tabu} to .7\textwidth {|X[c]|X[c]|X[c]|X[c]|X[c]|X[c]|X[c]|}
\hline
\gray{1} & \gray{4} & \textbf{6} & \gray{9} & \gray{15} & \gray{23} & \gray{24} \\
\hline
\end{tabu}
\end{center}
Mais $6\neq7$ donc 7 n'est pas dans le tableau.
\end{frame}

\begin{frame}[fragile]
\frametitle{Recherche dans un tableau trié (3)}
On coupe en deux à chaque fois $\Rightarrow \log_2(n)$ essais.

~

\textbf{Solution 2:} Recherche dichotomique, logarithmique $O(\log n)$
\begin{lstlisting}
bool isIn(int tab[], int n, int x)
{
    int left = 0, right = n-1;
    while (left <= right)
    {
        int middle = (left+right) / 2;
        if (x < tab[i]) right = middle - 1;
        else if (x > tab[i]) left = middle + 1;
        else return true;
    }
    return false;
}
\end{lstlisting}
\emph{Beaucoup} plus rapide!
\end{frame}

\begin{frame}
\frametitle{Limites pratiques}
Limites sur $n$ pour s'exécuter en quelques secondes:
\begin{center}
\begin{tabu}{lll}
    \toprule
    Complexité & Limite de $n$ & Exemple \\
    \midrule
    $O(1), O(\log n)$ & $\leq 10^{18}$ & (Taille limite d'un entier) \\
    $O(n)$ & $\leq$ 100\,M & Parcourir un tableau \\
    $O(n\log n)$ & $\leq$ 1\,M & Trier un tableau \\
    $O(n^2)$ & $\leq$ 10\,k & Boucle dans une boucle \\
    \bottomrule
\end{tabu}
\end{center}
En concours: regarder la deuxième colonne et cela donne la complexité.
\end{frame}

\end{document}
