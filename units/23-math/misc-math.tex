\documentclass[12pt]{beamer}
\usepackage{listings}
\beamertemplatenavigationsymbolsempty
\AtBeginSection[]
{
    \begin{frame}
    \frametitle{Table of Contents}
    \tableofcontents[currentsection]
    \end{frame}
}
\lstset{language=C++, basicstyle=\footnotesize, frame=single}

\title{Miscellaneous math}
\subtitle{Fast pow, Fibonacci, tortoise and hare}
\author{beCP Training}
\institute{\includegraphics[height=12em]{../shared-img/beoi-logo}}

\begin{document}

\frame{\titlepage}

\section{Fast pow}

\begin{frame}
\frametitle{Powers}
Definition:
\begin{itemize}
\item Chain multiplication
\item ``$n$-th power of $b$''
\item $b$ is the base, $n$ is the exponent
\item $b^n = \underbrace{b\times\cdots\times b}_\text{$n$ times}$
\end{itemize}
Examples:
\begin{itemize}
\item $3^0 = 1$ (by definition)
\item $3^1 = 3$
\item $3^2 = 3 \times 3 = 9$ (square)
\item $3^3 = 3 \times 3 \times 3 = 27$ (cube)
\end{itemize}
\end{frame}

\begin{frame}[fragile]
\frametitle{Power computation: linear}
\textbf{Problem:} compute the power the $n$-th power of $b$, for given $b$ and $n$.

~

\textbf{Solution 1:} Simple loop
\begin{lstlisting}[frame=single]
int nthPower(int b, int n)
{
    int power = 1;
    for (int i = 0; i < n; i++)
        power *= b;
    return power;
}
\end{lstlisting}
Complexity: $O(n)$
\end{frame}

\begin{frame}
\frametitle{Power computation: logarithmic (1)}
Can we do it faster? Yes, because associativity!

~

For example, to compute $3^{10}$, we can compute $3^5$ then square it:
\begin{itemize}
\item $3^2 = 3 \times 3 = 9$
\item $3^5 = 3^2 \times 3^2 \times 3 = 9 \times 9 \times 3 = 243$
\item $3^{10} = 3^5 \times 3^5 = 243 \times 243 = 59049$
\end{itemize}
Only 4 multiplications instead of 9.
\end{frame}

\begin{frame}[fragile]
\frametitle{Power computation: logarithmic (2)}
\textbf{Solution 2:} Recursive function
\begin{lstlisting}[frame=single]
int nthPower(int b, int n)
{
    // Initial case
    if (n == 0)
        return 1;
    
    // Recursive case
    int power = powerOfThree(b, n/2);
    power *= power;
    if (n % 2 == 1)
        power *= b;
    return power;
}
\end{lstlisting}
We divide $n$ by 2 on every call $\Rightarrow O(\log n)$
\end{frame}

\begin{frame}
\frametitle{Fast pow: usage}
When to use it:
\begin{itemize}
\item When linear time is too slow
\item Typically when computing a number of possibilities
\end{itemize}

~

Limits:
\begin{itemize}
\item Exponent $\leq 10^{18}$ if using \texttt{long long} (or more!)
\item Many powers with the same base $\Rightarrow$ store in an array
\item Be careful with overflows! Often, the statement asks for the result \emph{modulo} some number.
\end{itemize}
\end{frame}

\section{Matrix product}
\section{Fibonacci sequence}
\section{Tortoise and hare}

\end{document}
