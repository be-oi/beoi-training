% NOTE: This is just a dump of what I started for the introduction to algo before I realized fastpow was probably not the easiest thing to view first.

\begin{frame}[fragile]
\frametitle{Calcul de puissance: linéaire}
\textbf{Problème:} calculer la puissance $3^n = \underbrace{3\times\cdots\times3}_\text{$n$ fois}$

~

\textbf{Solution 1:} Une boucle
\begin{lstlisting}[frame=single]
int powerOfThree(int n)
{
    int power = 1;
    for (int i = 0; i < n; i++)
        power *= 3;
    return power;
}
\end{lstlisting}
Temps linéaire: $O(n)$
\end{frame}

\begin{frame}
\frametitle{Calcul de puissance: logarithmique (1)}
Peut-on faire plus efficace?

~

Oui! Par exemple, pour calculer $3^{10}$, on peut calculer $3^5$ puis le multiplier par lui-même:
\begin{itemize}
\item $3^2 = 3 \times 3 = 9$
\item $3^5 = 3^2 \times 3^2 \times 3 = 9 \times 9 \times 3 = 243$
\item $3^{10} = 3^5 \times 3^5 = 243 \times 243 = 59049$
\end{itemize}
Seulement 4 multiplications au lieu de 9.
\end{frame}

\begin{frame}[fragile]
\frametitle{Calcul de puissance: logarithmique (2)}
\textbf{Solution 2:} Une 
\begin{lstlisting}[frame=single]
int powerOfThree(int n)
{
    // Initial case
    if (n == 0)
        return 1;
    
    // Recursive case
    int power = powerOfThree(n/2);
    power *= power;
    if (n % 2 == 1)
        power *= 3;
    return power;
}
\end{lstlisting}
\end{frame}
