\frame{
    \frametitle{Z-algorithm}
    \framesubtitle{terminology}
    
    \begin{itemize}[<+->]
        \item Z-box = substring that matches with a prefix from the string
        \item Z-score $Z_i(S)$ = length of Z-box starting at index $i$
    \end{itemize}

    \begin{center}
        \only<3->{\includegraphics{img/z-box.png}}
    \end{center}

    \pause

    \begin{center}
    \only<4->{
    \begin{tabular}{|l|c|c|c|c|c|c|c|}
        \hline
        letter&A&A&B&A&A&A&B\\
        \hline
        Z-score&7&1&0&2&3&1&0\\
        \hline
    \end{tabular}}
    \end{center}
}

\frame{
    \frametitle{Z-algorithm}
    \framesubtitle{Matching}

    \begin{itemize}[<+->]
        \item $P$ = pattern
        \item $S$ = search string
        \item $\$$ = sentinel (not part of alphabet)
        \item return $i$ for each $i > 0$ where $Z_i(P\$S) = |P|$
    \end{itemize}
}

\frame{
    \frametitle{Z-algorithm}
    \framesubtitle{Calculating Z-scores}

    \begin{itemize}[<+->]
        \item Naive $\Rightarrow$ $O(n^2)$, possible in $O(n)$
        \item Keep track of the Z-box with right end furthest to the right (bounds: $[l, r]$)
        \item if current character in $[l, r]$: look at corresponding character in prefix (computed previously)
        \item expand if grows beyond $r$, update $[l, r]$
        \item else: calculate explicitely, update $[l, r]$
        \item (Nicely illustrated: https://www.cs.umd.edu/class/fall2011/cmsc858s/Lec02-zalg.pdf)
    \end{itemize}
}

\begin{frame}[fragile]
    \frametitle{Z-algorithm}
    \framesubtitle{code}

    \lstinputlisting[language=C++,basicstyle=\tiny,keywordstyle=\color{blue},firstline=4,lastline=28]{src/z-algo.cpp}
\end{frame}

\begin{frame}[fragile]
   \frametitle{Z-algorithm}
   \framesubtitle{code}

   \lstinputlisting[language=C++,basicstyle=\tiny,keywordstyle=\color{blue},firstline=29,lastline=44]{src/z-algo.cpp}
\end{frame}
