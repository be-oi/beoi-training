\subsection{Fibonacci sequence}

This is a simple example of how memorization can dramatically speed up
calculations.
We want to find the $n$-th Fibonacci number.
If we just implement the definition of the Fibonacci sequence, we get:
\begin{verbatim}
int fibo(int n)
{
    if(n==0)
        return 0;
    else if(n==1)
        return 1;
    else
        return fibo(n-1) + fibo(n-2);
}
\end{verbatim}
However, since \texttt{fibo(n)} calls both \texttt{fibo(n-1)} and
\texttt{fibo(n-2)}, the running time of that function will about double
every time $n$ increases by 1.
So the complexity is $O(2^n)$, which is quite crappy.

(Actually, it's more like $O(\phi^n)$, but it doesn't affect the crappiness.)

And that's because we compute the same thing \emph{many} times.
To avoid that, we just have to remember the solutions to the sub-problems,
that is, the previous Fibonacci numbers, in an array.
Then we can just fill up the array progressively, which is easier than
putting the results in throughout the algorithm as we find them.
\begin{verbatim}
int fibo(int n)
{
    vector<int> f;
    f.push_back(0);
    f.push_back(1);
    for(int i=2; i<=n; i++)
        f.push_back(f[i-1] + f[i-2]);
    return f[n];
}
\end{verbatim}
