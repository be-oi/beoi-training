\documentclass[12pt]{beamer}
\usepackage[utf8]{inputenc}
\usepackage[english]{babel}
\usepackage{listings}
\usepackage{tabu}
\usepackage{color}
\usepackage{booktabs}
\beamertemplatenavigationsymbolsempty
\AtBeginSection[]
{
    \begin{frame}
    \frametitle{Table of contents}
    \tableofcontents[currentsection]
    \end{frame}
}
\lstset{language=C++, basicstyle=\footnotesize, frame=single}
\newcommand{\gray}{\textcolor{gray}}

\title{Algorithms and complexity}
\subtitle{Definitions, big O notation}
\author{beOI Training}
\institute{\includegraphics[height=12em]{../share/beoi-logo}}

\begin{document}

\frame{\titlepage}

\section{Algorithms}

\begin{frame}
\frametitle{What's an algorithm?}
\begin{itemize}
\item A way to calculate a result
\item An idea to solve a problem
\item A series of instructions
\item The description of a programme
\end{itemize}
\end{frame}

\begin{frame}
\frametitle{What's a \emph{good} algorithm?}
For a normal programmer:
\begin{itemize}
\item Doesn't crash
\item Halts
\item Gives the right answer
\end{itemize}

~

For us:
\begin{itemize}
\item Is fast
\item Uses little memory
\item \textbf{Is accepted in competitions}
\end{itemize}
\end{frame}

\section{Complexity}

\begin{frame}
\frametitle{Measuring the efficiency}
Ideas:
\begin{itemize}
\item Timing
\item Measure the RAM usage
\end{itemize}

~

But this varies depending on:
\begin{itemize}
\item The language
\item The implementation
\item The machine
\item The time of the day
\end{itemize}
\end{frame}

\begin{frame}
\frametitle{Big O notation}

Finding an \emph{intrinsic} measure of efficiency:
\begin{itemize}
\item Grow the input
\item Observe how the execution speed changes
\end{itemize}

~

Example: calculating $1+\cdots+n$. If $n$ is multiplied by 2, the execution time:
\begin{itemize}
\item Stays constant: $O(1)$
\item Is multiplied by 2: $O(n)$
\item Is multiplied by 4: $O(n^2)$
\end{itemize}

~

It doesn't depend on the constant factors!
\end{frame}

\begin{frame}[fragile]
\frametitle{Constant execution time}
\textbf{Task:} compute the sum $1+2+\cdots+n$.

~

\textbf{Solution 1:} A simple calculation
\begin{lstlisting}
int sum = n * (n+1) / 2;
\end{lstlisting}
\begin{itemize}
\item Execution time doesn't change if $n$ doubles
\item ``Constant'' execution time
\item $O(1)$ complexity
\item Execution time ``proportional to $1$''
\end{itemize}
\end{frame}

\begin{frame}[fragile]
\frametitle{Linear execution time}
\textbf{Solution 2:} A loop
\begin{lstlisting}
int sum = 0;
for (int i = 1; i <= n; i++)
    sum += i;
\end{lstlisting}
\begin{itemize}
\item Execution time doubles if $n$ doubles
\item ``Linear'' execution time
\item $O(n)$ complexity
\item Execution time ``proportional to $n$''
\end{itemize}
\end{frame}

\begin{frame}[fragile]
\frametitle{Quadratic execution time}
\textbf{Solution 3:} Two nested loops (useless!)
\begin{lstlisting}
int sum = 0;
for (int i = 1; i <= n; i++)
    for (int j = 1; j <= i; j++)
        sum++;
\end{lstlisting}
\begin{itemize}
\item Execution time quadruples if $n$ doubles
\item ``Quadratic'' execution time
\item $O(n^2)$ complexity
\item Execution time ``proportional to $n^2$''
\end{itemize}
\end{frame}

\begin{frame}
\frametitle{Powers}
Definition:
\begin{itemize}
\item Successive multiplications
\item ``3 to the power of $n$''
\item $3^n = \underbrace{3\times\cdots\times3}_\text{$n$ times}$
\end{itemize}
Examples:
\begin{itemize}
\item $3^0 = 1$ (by definition)
\item $3^1 = 3$
\item $3^2 = 3 \times 3 = 9$ (squared)
\item $3^3 = 3 \times 3 \times 3 = 27$ (cubed)
\end{itemize}
\end{frame}

\begin{frame}
\frametitle{Logarithms: intuition (1)}
Game with two players:
\begin{itemize}
\item Alice chooses a number between 1 and 16
\item During each turn:
    \begin{itemize}
    \item Bob gives one or more numbers to Alice
    \item Alice says if her chosen number is among those given by Bob
    \end{itemize}
\item Bob wins when he guesses the correct number
\item How can he win in as few turns as possible?
\end{itemize}
\end{frame}

\begin{frame}
\frametitle{Logarithms: intuition (2)}
\textbf{Strategy:} Give a list with half the total available numbers
\begin{itemize}
\item First give 8 numbers among the 16
\item Then 4 among the 8 remaining
\item Then 2 among the 4 remaining
\item Then 1 among the 2 remaining
\item Found!
\end{itemize}

~

Thus, 4 questions suffice.
\end{frame}

\begin{frame}
\frametitle{Logarithms: intuition (3)}
Generally, if we start with $n$ numbers, how may questions needed?

~

How many times can we cut in half?
\begin{itemize}
\item If $n=2$, once
\item If $n=4$, twice
\item If $n=8$, three times
\item If $n=16$, four times
\end{itemize}
\end{frame}

\begin{frame}
\frametitle{Logarithm in base 2}
The function that answers that question is $\log_2$: the logarithm in base 2. For example
\begin{itemize}
\item $\log_2 (2) = 1$
\item $\log_2 (4) = 2$
\item $\log_2 (8) = 3$
\item $\log_2 (16) = 4$
\end{itemize}
Bob's strategy is $O(\log_2(n)) = O(\log n)$.

~

In other words, the logarithm is the exponent we have to put on the 2 to reach $n$:
\[ x = \log_2 (n)\ \Leftrightarrow\ 2^x = n \]
\end{frame}

\begin{frame}
\frametitle{General logarithm (bonus)}
This can also apply for numbers other than 2! The logarithm in base $a$, is the number of times we can divide by $a$. For example:
\begin{itemize}
\item $\log_3(27) = 3$
\item $\log_4(16) = 2$
\item $\log_5(5) = 1$
\end{itemize}

~

In other words, it's the exponent we have to put on the $a$ to reach $n$:
\[ x = \log_a (n)\ \Leftrightarrow\ a^x = n \]
It's sort of the ``inverse'' of powers.
\end{frame}

\begin{frame}[fragile]
\frametitle{Searching in a sorted array (1)}
We are given a sorted array (numbers in increasing order):
\begin{center}
\def\arraystretch{1.3}
\begin{tabu} to .7\textwidth {|X[c]|X[c]|X[c]|X[c]|X[c]|X[c]|X[c]|}
\hline
1 & 4 & 6 & 9 & 15 & 23 & 24 \\
\hline
\end{tabu}
\end{center}
Our task is to check if $x$ is in the array.

~

\textbf{Solution 1:} Go through all the elements: linear, $O(n)$
\begin{lstlisting}
bool isIn(int tab[], int n, int x)
{
    for (int i = 0; i < n; i++)
        if (tab[i] == x)
            return true;
    return false;
}
\end{lstlisting}
\end{frame}

\begin{frame}
\frametitle{Searching in a sorted array (2)}
We are searching for 7.

~

\textbf{Idea:} check the middle element and compare:
\begin{center}
\def\arraystretch{1.3}
\begin{tabu} to .7\textwidth {|X[c]|X[c]|X[c]|X[c]|X[c]|X[c]|X[c]|}
\hline
1 & 4 & 6 & \textbf{9} & 15 & 23 & 24 \\
\hline
\end{tabu}
\end{center}
Too big ($9>7$), go to the left:
\begin{center}
\def\arraystretch{1.3}
\begin{tabu} to .7\textwidth {|X[c]|X[c]|X[c]|X[c]|X[c]|X[c]|X[c]|}
\hline
1 & \textbf{4} & 6 & \gray{9} & \gray{15} & \gray{23} & \gray{24} \\
\hline
\end{tabu}
\end{center}
To small ($4<7$), go to the right:
\begin{center}
\def\arraystretch{1.3}
\begin{tabu} to .7\textwidth {|X[c]|X[c]|X[c]|X[c]|X[c]|X[c]|X[c]|}
\hline
\gray{1} & \gray{4} & \textbf{6} & \gray{9} & \gray{15} & \gray{23} & \gray{24} \\
\hline
\end{tabu}
\end{center}
Only one element left, but $6\neq7$: hence, 7 is not in the array
\end{frame}

\begin{frame}[fragile]
\frametitle{Searching in a sorted array (3)}
Each time, we cut in half $\Rightarrow \log_2(n)$ tries.

~

\textbf{Solution 2:} Dichotomous search: logarithmic, $O(\log n)$
\begin{lstlisting}
bool isIn(int tab[], int n, int x)
{
    int left = 0, right = n-1;
    while (left <= right)
    {
        int mid = (left+right) / 2;
        if (x < tab[mid]) right = mid - 1;
        else if (x > tab[mid]) left = mid + 1;
        else return true;
    }
    return false;
}
\end{lstlisting}
\emph{Way} faster!
\end{frame}

\begin{frame}
\frametitle{Limits in practice}
Limits on $n$ so that the program can execute in a few seconds:
\begin{center}
\begin{tabu}{lll}
    \toprule
    Complexity & Limit on $n$ & Example \\
    \midrule
    $O(1), O(\log n)$ & $\leq 10^{18}$ & (Maximal integer size) \\
    $O(n)$ & $\leq$ 100\,M & Going through an array \\
    $O(n\log n)$ & $\leq$ 1\,M & Sorting an array \\
    $O(n^2)$ & $\leq$ 10\,k & Loop nested in a loop \\
    \bottomrule
\end{tabu}
\end{center}
For contests: find the limit on $n$ in the second column, and get the complexity in the first column.
\end{frame}

\end{document}